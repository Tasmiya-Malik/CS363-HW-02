\documentclass[a4paper]{exam}

\usepackage{amsmath, amsfonts}
\usepackage{hyperref}

\printanswers

\title{CS363 Networks \& Collective Behavior - Homework 2}
\author{Tasmiya Malik tm06183, Ifrah Ilyas}

\begin{document}
\maketitle
\begin{questions}
        
    \question There is a question here.
    \begin{solution}
        From the question we have,
        $$n=1\times 10^9 \hspace{5cm} p=1\times 10^{-5}$$
        \begin{parts}
            \part $$<m> = n\textbf{C}2\cdot p$$
            $$<m> = 1\times 10^9\textbf{C}2\cdot 1\times 10^{-5}$$
            $$<m> = 4.99\times 10^{12}$$

            \part $$<k> = (n-1)\cdot p$$
            $$<k>=(1\times 10^9 - 1)\cdot 1\times 10^{-5}$$
            $$<k> = 9999.99$$

            \part $$<k> = (n-1)\cdot p$$
            $$1\times 10^5 = (1\times 10^9 - 1)\cdot p$$
            $$p = \frac{1\times 10^5}{1\times 10^9 - 1} = 1\times 10^{-4}$$

            \part $$APL = \frac{\ln (n)}{\ln (<k>)}$$
            $$\frac{\ln (1\times 10^9)}{\ln (9999.99)} = 2.25$$

            \part No, all the nodes will not be a part of the giant component. This is because the probability $p$ is near to $0$ rather than $1$, thus, not all nodes will be connected.
        \end{parts}
    \end{solution}
\end{questions}
\end{document}

